%-------------------------------------------------------------------------------
\section{Introduction}
%-------------------------------------------------------------------------------
The estimation of finite-horizon discrete choice dynamic programming (DCDP) models has generated valuable insights in diverse subfields of economics such as industrial organization, labor economics, and marketing.\footnote{See \citet{Keane.2011d} for a recent survey of the literature.} DCDP models are structural economic models as they make explicit the agents' objective as well as the informational and institutional constraints under which they operate. Thus they allow to assess the relative importance of competing economic mechanisms that guide agents' decision making and permit the ex ante evaluation of alternative policy proposals \citep{Wolpin.2013}. In these models, economic agents make repeated choices over multiple periods. They are forward-looking and thus take the future consequences of their immediate actions into account. However, agents operate in an uncertain economic environment as at least parts of their future payoffs are unknown at the time of their decision.\\\newline
%
Estimating a finite-horizon DCDP model poses computational challenges as it requires the repeated solution of a dynamic programming (DP) problem under uncertainty by backward induction. The well known curse of dimensionality \citep{Bellman.1957} is a major impediment to the application of more realistic models and their verification and validation. To alleviate the computational burden, \citet{Keane.1994} propose to work with an approximate solution to the dynamic programming problem instead. They suggest to solve the DP problem during the backward induction procedure at only a subset of states in each period and simply use interpolated values for all other states. This introduces approximation error and requires a careful assessment of the reliability of results. \citet{Keane.1994} provide some very encouraging Monte Carlo evidence for a prototypical model of occupational choice.\\\newline
%
I will describe their approach in detail, successfully recompute their original quality diagnostics, and provide some additional insights that underscore the trade-off between computation time and the accuracy of estimation results. The rest of the paper is structured as follows. In Section \ref{Setup}, I present the economics of the basic model and outline the approach to its solution and estimation. I then turn to the results of my recomputation in Section \ref{Approximation} and provide some additional diagnostics regarding the reliability of the proposed interpolation scheme. Section \ref{Conclusion} concludes. This manuscript is supplemented by an open-source Python package for the simulation and estimation of a prototypical discrete choice dynamic programming model. The package is available at \url{http://respy.readthedocs.io}.
