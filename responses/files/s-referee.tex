\newpage\section*{Responses to Referee}\vspace{1.0cm}
%-------------------------------------------------------------------------------
%-------------------------------------------------------------------------------
\begin{boenumerate}
%-------------------------------------------------------------------------------
%	Comment 1
%-------------------------------------------------------------------------------
\item \textit{First, the KW paper can be thought of as consisting of two parts: First, an analysis of the accuracy of the approximate solution in terms of the value functions and policy functions. Second, an analysis of how using approximate solutions for DCDP models affects parameter estimates when those models are estimated. The part of the present paper that deals with
accuracy of the approximate solution seems fine, but the part of the paper that deals with estimation (starting on the middle of page 8) is very confusing.}\vspace{0.5cm}

Thank you for this comment.\vspace{0.50cm}
%-------------------------------------------------------------------------------
%	Comment 2
%-------------------------------------------------------------------------------
\item \textit{The main problem is that the author never writes out the choice probabilities or the likelihood function for the model, and he doesn't explain how they are simulated. According to KW, they use 200 draws and a kernel smoothing algorithm to smooth the likelihood. But the author doesn't discuss what he does here. This is critical, because his main finding is that there are local maxima problems in the simulated likelihood. But this may well be because of how the tuning parameter is set in the kernel smoothing algorithm. This needs to be discussed. One proposal that has been made by KW in other work is to start with a large bandwidth and then make it smaller as you approach the optimum (or, alternatively, as the author states here, to increase the number of draws as one approaches the optimum).}\vspace{0.5cm}

Thank you for this comment.\vspace{0.50cm}
%-------------------------------------------------------------------------------
%	Comment 3
%-------------------------------------------------------------------------------
\item \textit{Another limitation of the paper is that it misses a great opportunity to truly update the KW results. Of course computers are vastly faster now than they were in 1994. I assume this means that it should be possible to approximate DP solutions much more accurately than KW did back in 1994. For example, KW presumably report computation times for their exact and approximate solutions. How much faster are these times now? Conversely, if one were willing to solve the problem in the same time as KW, how much more accurate could you make the solution today (by using more draws and/or state points)? Or, one could ask how much bigger of a problem could be solved today to the same accuracy and computation time. One easy way to
look at this might be to expand the state space by increasing the time horizon (T).}\vspace{0.5cm}

I very much agree that it is useful to assess the contribution of \citet{Keane.1994} in its historical context. Thus, I added the proposed extension to Appendix \ref{Computational Details}.
%-------------------------------------------------------------------------------
%-------------------------------------------------------------------------------s
\end{boenumerate}
