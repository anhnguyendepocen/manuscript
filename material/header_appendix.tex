\documentclass[a4paper,12pt,bold]{scrartcl}

\renewcommand{\baselinestretch}{1.3}\normalsize
\newcommand{\vect}[1]{\mathbf{#1}}
\newcommand{\thin}{\thinspace}
\newcommand{\thick}{\thickspace}

\newcommand{\N}{\mathcal{N}}	%Normal Distribution
\newcommand{\U}{\mathrm{U}}	%Uniform Distribution
\newcommand{\D}{\mathrm{D}}	%Dirichlet Distribution
\newcommand{\W}{\mathrm{W}}	%Wishart Distribution
\newcommand{\E}{\mathbb{E}}		%Expectation
\newcommand{\Iden}{\mathbb{I}}	%Identity Matrix
\newcommand{\Ind}{\mathrm{I}}	%Indicator Function

\newcommand{\Prob}{\mathrm{Prob}}

\newcommand{\bs}{\boldsymbol}
\newcommand{\var}{\mathrm{var}\thin}
\newcommand{\plim}{\mathrm{plim}\thin}
\newcommand{\cov}{\mathrm{cov}\thin}
\newcommand\indep{\protect\mathpalette{\protect\independenT}{\perp}}
\def\independenT#1#2{\mathrel{\rlap{$#1#2$}\mkern5mu{#1#2}}}
\usepackage{chngcntr}

\usepackage{relsize}
%\usepackage{algpseudocode}
%\usepackage[]{algorithm2e}
\usepackage{epigraph}
\usepackage[sans]{dsfont}
\usepackage[round,longnamesfirst]{natbib}
\usepackage{bm}
\usepackage{lipsum}													%matrix symbol
\usepackage{setspace}																						%Fu�noten (allgm.
\usepackage{hyperref}
%Zeilenabst�nde)
\usepackage{threeparttable}
\usepackage{lscape}																					\usepackage{listings}		%Querformat
\usepackage[latin1]{inputenc}																		%Umlaute
\usepackage{graphicx}
\usepackage{amsmath}
\usepackage{amssymb}
\usepackage{fancybox}																						%Boxen und Rahmen
\usepackage{appendix}
\usepackage{enumerate}
                                                                                %EURO Symbol
\usepackage{tabularx}
\usepackage{longtable}
\usepackage{subfig,float}																				%Mehrseitige Tabellen
\usepackage{color,colortbl}																			%Farbige Tabellen
\usepackage[left=3cm, right=3cm, top=3cm, bottom=3cm]{geometry}
%\usepackage[normal]{caption2}[2002/08/03]												%Titel ohne float - Umgebung
\definecolor{lightgrey}{gray}{0.95}	%Farben mischen
\definecolor{grey}{gray}{0.85}
\definecolor{darkgrey}{gray}{0.80}

\newcommand{\mc}{\multicolumn}
\parindent0pt

\usepackage{tikz}
\usetikzlibrary{positioning}

\usepackage{caption}
\captionsetup[figure]{labelfont=bf}
\captionsetup[table]{skip=15pt}


\newtheorem{Definition}{Definition}
\newtheorem{Remark}{Remark}
\newtheorem{Lemma}{Lemma}
\newtheorem{Theorem}{Theorem}
\newtheorem{Excercise}{Excercise}
\newtheorem{Result}{Result}
\newtheorem{Proposition}{Proposition}
\newtheorem{Prediction}{Prediction}
\newtheorem{Solution}{Solution}
\newtheorem{Problem}{Problem}

\setlength{\skip\footins}{1.0cm}
\deffootnote[1em]{1.1em}{0em}{\textsuperscript{\thefootnotemark}}
\renewcommand{\arraystretch}{1.05}
\usepackage[labelfont=bf]{caption}


\makeatletter
\newenvironment{manquotation}[2][2em]
  {\setlength{\@tempdima}{#1}%
   \def\chapquote@author{#2}%
   \parshape 1 \@tempdima \dimexpr\textwidth-2\@tempdima\relax%
   \itshape}
  {\par\normalfont\hfill--\ \chapquote@author\hspace*{\@tempdima}\par\bigskip}
\makeatother

\newenvironment{boenumerate}
{\begin{enumerate}\renewcommand\labelenumi{\textbf{(\theenumi)}}}
{\end{enumerate}}
\usepackage[english]{babel}

\usepackage{booktabs}
\usepackage{xr}

\captionsetup{belowskip=12pt,aboveskip=4pt}

\onehalfspacing
