%-------------------------------------------------------------------------------
\section{Recomputation Instructions}
%-------------------------------------------------------------------------------
I provide an image of a virtual machine (VM) for download to ensure full recomputability of my results. The image contains a software-based emulation of a computer, where all the required software is already pre-installed. This makes recomputation straightforward.\\\newline
%
Two additional software tools are required: (1)  \texttt{VirtualBox} and (2) \texttt{Vagrant}. \texttt{VirtualBox} is a virtualization software, while \texttt{Vagrant} provides a convenient wrapper around it. Both are free and open-source. Please consult their websites for installation instructions. The following instructions were tested for \texttt{VirtualBox 5.0} and \texttt{Vagrant 1.8}.\newline

Once \texttt{VirtualBox} and \texttt{Vagrant} are available, the image can be downloaded and accessed by the following commands:

\vspace{0.2cm}\begin{lstlisting}[language=bash]
    $ vagrant init structRecomputation/base
    $ vagrant up --provider virtualbox
    $ vagrant ssh
\end{lstlisting}\vspace{0.2cm}

As all the required software is already installed, recomputation is straightforward. Simply typing the following command into the terminal produces all the results in the paper:
\vspace{0.2cm}\begin{lstlisting}[language=bash]
    $ ./recompute
\end{lstlisting}\vspace{0.2cm}

The output files will be available in the \verb+_published+ subdirectory. This process takes a couple of days due to the large number of Monte Carlo iterations for the initial Monte Carlo exercise. There is a slight difference in the order and sign of the coefficients between the output files and the results in this paper, please see  \verb+respy+'s online documentation for details. Table \ref{Mapping} provides the mapping between the output files and the results reported in the two relevant publications.

\begin{table}\onehalfspacing
\begin{center}
\begin{threeparttable}
  \caption{Mapping between Files and Results}
  \label{Mapping}
  \begin{tabular}{lcc}\toprule
   File & \mc{1}{c}{\cite{Keane.1994}} &  \cite{Eisenhauer.2017}\\
   \midrule
   \mc{3}{c}{Correct Choices} \\
   \midrule
   \verb+table_2.1.txt+ & Table 2.1 & Table \ref{Correct Choices: One} \\
   \verb+table_2.2.txt+ & Table 2.2 & Table \ref{Correct Choices: Two} \\
   \verb+table_2.3.txt+ & Table 2.3 & Table \ref{Correct Choices: Three} \\
   \midrule
   \mc{3}{c}{Monte Carlo Estimation} \\
   \midrule
   \verb+table_4.1.txt+ & Table 4.1 & Table \ref{Monte Carlo: One}  \\
   \verb+table_4.2.txt+ & Table 4.2 & Table \ref{Monte Carlo: Two}  \\
   \verb+table_4.3.txt+ & Table 4.3 & Table \ref{Monte Carlo: Three} \\
   \midrule
   \mc{3}{c}{Choice Patterns} \\
   \midrule
   \verb+choices_one.png+   & --- & Figure \ref{Choice Patterns} \\
   \verb+choices_two.png+   & --- & Figure \ref{Choice Patterns} \\
   \verb+choices_three.png+ & --- & Figure \ref{Choice Patterns} \\
   \midrule
   \mc{3}{c}{Interpolation Schemes} \\
   \midrule
   \verb+schemes.txt+   & --- & Table \ref{Interpolation Schemes} \\
  \bottomrule
  \end{tabular}\scriptsize
\end{threeparttable}
\end{center}
\end{table}

