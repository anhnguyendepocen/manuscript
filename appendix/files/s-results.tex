%-------------------------------------------------------------------------------
\section{Additional Results}\label{Results}
%-------------------------------------------------------------------------------
This section presents all my results for each of the parameterizations in Table \ref{Parameterizations}. The \textit{exact solution} is constructed using 100,000 random draws for the evaluation of $E\max$ at all states at the true parameter values. The \textit{exact sample} refers to a set of 1,000 simulated agents based on the \textit{exact solution}. As on overall measure of the approximation error, I use the root-mean-square error (RMSE) by comparing the choice probabilities in the \textit{exact sample} to a newly simulated set of 1,000 agents based on the relevant alternative parameterization of the model.
%-------------------------------------------------------------------------------
\subsection{Parameterizations}
%-------------------------------------------------------------------------------
\begin{center}
\begin{threeparttable}
  \caption{Parameterizations}
  \label{Parameterizations}
  \begin{tabular}{crrr}\toprule
  \mc{1}{c}{Parameter} & \mc{1}{r}{Data One} & \mc{1}{r}{Data Two} & \mc{1}{r}{Data Three}  \\
  \midrule
  $\alpha_{10}$         &     9.2100 &     9.2100 &     8.0000 \\
  $\alpha_{11}$         &     0.0380 &     0.4000 &     0.0700 \\
  $\alpha_{12}$         &     0.0330 &     0.0330 &     0.0550 \\
  $\alpha_{13}$         &     0.0005 &     0.0005 &     0.0000 \\
  $\alpha_{14}$         &     0.0000 &     0.0000 &     0.0000 \\
  $\alpha_{15}$         &     0.0000 &     0.0000 &     0.0000 \\
  $\alpha_{20}$         &     8.4800 &     8.2000 &     7.9000 \\
  $\alpha_{21}$         &     0.0700 &     0.0800 &     0.0700 \\
  $\alpha_{22}$         &     0.0670 &     0.0670 &     0.0600 \\
  $\alpha_{23}$         &     0.0010 &     0.0010 &     0.0000 \\
  $\alpha_{24}$         &     0.0220 &     0.0220 &     0.0550 \\
  $\alpha_{25}$         &     0.0005 &     0.0005 &     0.0000 \\
  $\beta_{0}$           &     0.0000 &  5,000.0000 &  5,000.0000 \\
  $\beta_{1}$           &     0.0000 &  5,000.0000 &  5,000.0000 \\
  $\beta_{2}$           &  4,000.0000 & 15,000.0000 & 20,000.0000 \\
  $\gamma_{0}$          & 17,750.0000 & 14,500.0000 & 21,500.0000 \\
  $(\sigma_{11})^{1/2}$ &     0.2000 &     0.4000 &     1.0000 \\
  $\sigma_{12}$         &     0.0000 &     0.0000 &     0.5000 \\
  $\sigma_{13}$         &     0.0000 &     0.0000 &     0.0000 \\
  $\sigma_{14}$         &     0.0000 &     0.0000 &     0.0000 \\
  $(\sigma_{22})^{1/2}$ &     0.2500 &     0.5000 &     1.0000 \\
  $\sigma_{23}$         &     0.0000 &     0.0000 &     0.0000 \\
  $\sigma_{24}$         &     0.0000 &     0.0000 &     0.0000 \\
  $(\sigma_{33})^{1/2}$ &  1,500.0000 &  6,000.0000 &  7,000.0000 \\
  $\sigma_{34}$         &     0.0000 &     0.0000 & $-2.975\times10^7$\\
  $(\sigma_{44})^{1/2}$ &  1,500.0000 &  6,000.0000 &  8,500.0000 \\
  \bottomrule
  \end{tabular}
  %\scriptsize
  %\begin{tablenotes}\item \textbf{Notes:}
  %\end{tablenotes}
\end{threeparttable}
\end{center}\vspace{0.5cm}

%-------------------------------------------------------------------------------
\subsection{Choice Patterns}
%-------------------------------------------------------------------------------
Figure \ref{Choice Patterns} shows the share of agents in the \textit{exact sample} opting for each of the four alternatives by period.\\\newline
\begin{figure}[h!]
\caption{Choice Patterns}\label{Choice Patterns}
\centering
\subfloat[Data One]{
\scalebox{0.30}{\includegraphics{../material/graph_patterns_one}}}\vspace{0.5cm}
\subfloat[Data Two]{
\scalebox{0.30}{\includegraphics{../material/graph_patterns_two}}}\vspace{0.5cm}
\subfloat[Data Three]{
\scalebox{0.30}{\includegraphics{../material/graph_patterns_three}}} 
\begin{center}
%\begin{minipage}[t]{0.85\columnwidth}\vspace{-0.25cm}
%\item\scriptsize{\textbf{Notes:} }
%\end{minipage}
\end{center}
\end{figure}
\clearpage
%-------------------------------------------------------------------------------
\subsection{Correct Choices}
%-------------------------------------------------------------------------------
Tables \ref{Correct Choices: One} - \ref{Correct Choices: Three} show the proportion of correct choices for alternative interpolation schemes.
\begin{table}[p]\onehalfspacing
\begin{center}
\begin{threeparttable}
  \caption{Correct Choices, Dataset One}
  \label{Correct Choices: One}
  \begin{tabular}{lrrrrr}\toprule
  Points     & All & All & All   & 2,000 & 500   \\
  $E\max$ Draws & 2,000 & 1,000 & 250 & 2,000 & 2,000  \\
  \midrule
  \mc{6}{c}{At Selected Periods} \\
  \midrule
  Period & \mc{5}{c}{} \\
  \phantom{1}1      &  1.000 &  0.998 &  0.938 &  0.967 &  0.942 \\
  10                &  0.990 &  1.000 &  0.989 &  0.988 &  0.979 \\
  20                &  1.000 &  1.000 &  1.000 &  0.998 &  0.999 \\
  30                &  1.000 &  1.000 &  1.000 &  0.994 &  0.998 \\
  40                &  1.000 &  1.000 &  1.000 &  1.000 &  1.000 \\
  Total             &  0.998 &  0.999 &  0.994 &  0.993 &  0.991 \\ % in table 0.9989 and 0.9938
  \midrule
  \mc{6}{c}{Number of Periods over the Lifetime} \\
  \midrule
  Periods & \mc{5}{c}{} \\
  \phantom{0}1 - 10  &   0.000 &  0.000 &  0.000 &  0.000 &  0.000 \\
  11 - 35            &   0.000 &  0.000 &  0.000 &  0.000 &  0.000 \\
  36 - 38            &   0.010 &  0.000 &  0.020 &  0.027 &  0.046 \\
  39                 &   0.036 &  0.043 &  0.181 &  0.190 &  0.252 \\
  40                 &   0.963 &  0.957 &  0.799 &  0.783 &  0.702 \\
  Average            &  39.962 & 39.957 & 39.777 & 39.752 & 39.649 \\
  \bottomrule
  \end{tabular}
\end{threeparttable}
\end{center}
\end{table}

\begin{table}[p]\onehalfspacing
\begin{center}
\begin{threeparttable}
  \caption{Correct Choices, Dataset Two}
  \label{Correct Choices: Two}
  \begin{tabular}{lrrrrr}\toprule
  Points     & All & All & All   & 2,000 & 500   \\
  $E\max$ Draws & 2,000 & 1,000 & 250 & 2,000 & 2,000  \\
  \midrule
  \mc{6}{c}{At Selected Periods} \\
  \midrule
  Period & \mc{5}{c}{} \\
  \phantom{1}1      &  0.998 &  0.994 &  0.993 &  0.996 &  0.988 \\
  10                &  1.000 &  0.998 &  0.995 &  0.990 &  0.972 \\
  20                &  1.000 &  0.997 &  0.994 &  0.979 &  0.961 \\
  30                &  0.998 &  1.000 &  0.998 &  0.988 &  0.989 \\
  40                &  1.000 &  1.000 &  1.000 &  1.000 &  1.000 \\
  Total             &  0.998 &  0.997 &  0.995 &  0.990 &  0.981 \\
  \midrule
  \mc{6}{c}{Number of Periods over the Lifetime} \\
  \midrule
  Periods & \mc{5}{c}{} \\
  \phantom{1}1 - 10 &  0.000 &  0.000 &  0.000 &  0.000 &  0.000 \\
  11 - 35           &  0.000 &  0.000 &  0.000 &  0.001 &  0.001 \\
  36 - 38           &  0.003 &  0.003 &  0.012 &  0.062 &  0.157 \\
  39                &  0.040 &  0.085 &  0.172 &  0.260 &  0.361 \\
  40                &  0.957 &  0.912 &  0.816 &  0.677 &  0.481 \\
  Average           & 39.954 & 39.909 & 39.804 & 39.600 & 39.265 \\
  \bottomrule
  \end{tabular}
\end{threeparttable}
\end{center}
\end{table}

\begin{table}[p]\onehalfspacing
\begin{center}
\begin{threeparttable}
  \caption{Correct Choices, Dataset Three}
  \label{Correct Choices: Three}
  \begin{tabular}{lrrrrr}\toprule
  Points     & All & All & All   & 2,000 & 500   \\
  $E\max$ Draws & 2,000 & 1,000 & 250 & 2,000 & 2,000  \\
  \midrule
  \mc{6}{c}{At Selected Periods} \\
  \midrule
  Period & \mc{5}{c}{} \\
  \phantom{1}1      &  0.995 &  0.993 &  0.985 &  0.991 &  0.979 \\
  10                &  0.995 &  0.995 &  0.982 &  0.975 &  0.931 \\
  20                &  0.995 &  0.997 &  0.994 &  0.979 &  0.940 \\
  30                &  0.994 &  0.999 &  0.989 &  0.974 &  0.972 \\
  40                &  1.000 &  1.000 &  1.000 &  1.000 &  1.000 \\
  Total             &  0.995 &  0.995 &  0.991 &  0.980 &  0.959 \\ % 0.9957 /- / 0.9918  / 0.9807 / 
  \midrule
  \mc{6}{c}{Number of Periods over the Lifetime} \\
  \midrule
  Periods & \mc{5}{c}{} \\
  \phantom{1}1 - 10 &  0.000 &  0.000 &  0.000 &  0.000 &  0.000 \\
  11 - 35           &  0.000 &  0.000 &  0.000 &  0.003 &  0.030 \\
  36 - 38           &  0.015 &  0.015 &  0.038 &  0.187 &  0.432 \\
  39                &  0.142 &  0.150 &  0.249 &  0.324 &  0.304 \\
  40                &  0.843 &  0.835 &  0.713 &  0.486 &  0.234 \\
  Average           & 39.827 & 39.817 & 39.671 & 39.226 & 38.374 \\
  \bottomrule
  \end{tabular}
\end{threeparttable}
\end{center}\end{table}

%-------------------------------------------------------------------------------
\subsection{Monte Carlo Exercise}
%-------------------------------------------------------------------------------
Tables \ref{Monte Carlo: One} - \ref{Monte Carlo: Three} show the estimation performance for each of the model parameters during the initial Monte Carlo exercise. Let $\theta_i$ denote the true value of parameter $i$, $\hat{\theta}_i$ its average estimate across all bootstrap replications, and $\hat{\theta}_{ij}$ the estimated parameter in iteration $j$. The statistics in the Table \ref{Monte Carlo: One} - \ref{Monte Carlo: Three} are calculated as follows:

\renewcommand\arraystretch{2}
\begin{align*}\begin{array}{l@{\qquad}l}
\text{Bias} & \hat{\theta}_i - \theta_i\\\medskip
\text{$t$ - statistic} & \left(\frac{\hat{\theta}_i - \theta_i}{\sigma_{\hat{\theta_i}}}\right) \sqrt{40} \\
\text{Standard Deviation} & \left[ \frac{1}{39} \sum^{40}_{j = 1} (\hat{\theta}_{ij} - \hat{\theta}_i)^2
\right]^{\tfrac{1}{2}}
\end{array}
\end{align*}
\renewcommand\arraystretch{1}

Note that the table contains the Cholesky decomposition parameters $a_{ij}$ of the covariance matrix of the shocks to the immediate rewards. I report the RMSE, the total number of evaluations of the criterion function, and the number of steps of the optimizer as their average across all 40 bootstrap iterations.\\\newline
%
I specify 200 interpolation points, use 500 random draws for the evaluation of $E\max$, and allow for a maximum of 1,000 evaluations of the criterion function by the optimizer during each estimation.\clearpage

\begin{table}\onehalfspacing
\begin{center}
\begin{threeparttable}
  \caption{Monte Carlo Exercise, Dataset One}
  \label{Monte Carlo: One}
  \begin{tabular}{crrrr}\toprule

  Parameter & True Value & Bias & $t$ - statistic & Std. Deviation \\
  \midrule
  $\alpha_{10}$ &     \phantom{20000}9.2100 &    \phantom{-17}0.0012 &     1.6744 &      0.0045 \\
  $\alpha_{11}$ &     0.0380 &      0.0000 &      0.5024 &       0.0006 \\
  $\alpha_{12}$ &     0.0330 &     -0.0002 &     -2.8043 &       0.0003 \\
  $\alpha_{13}$ &     0.0005 &      0.0000 &     12.4900 &       0.0000 \\
  $\alpha_{14}$ &     0.0000 &     -0.0010 &     -4.8287 &       0.0013 \\
  $\alpha_{15}$ &     0.0000 &      0.0000 &      2.1206 &       0.0001 \\
  $\alpha_{20}$ &     8.4800 &     -0.0007 &     -1.2986 &       0.0032 \\
  $\alpha_{21}$ &     0.0700 &     -0.0000 &     -2.1280 &       0.0001 \\
  $\alpha_{22}$ &     0.0670 &     -0.0002 &     -2.3279 &       0.0005 \\
  $\alpha_{23}$ &     0.0010 &      0.0000 &      3.5786 &       0.0000 \\
  $\alpha_{24}$ &     0.0220 &      0.0000 &      2.1136 &       0.0001 \\
  $\alpha_{25}$ &     0.0005 &      0.0000 &      0.5725 &       0.0000 \\
  $\beta_{0}$   &     0.0000 &    -91.0443 &     -4.0973 &     140.5369 \\
  $\beta_{1}$   &     0.0000 &     -7.9753 &     -0.3319 &     151.9692 \\
  $\beta_{2}$   &  4,000.0000 &   -12.2382 &     -0.3042 &     254.4796 \\
  $\gamma_{0}$  & 17,750.0000 &   -60.7969 &     -1.5008 &     256.1998 \\
  $a_{11}$      &     0.2000 &     -0.0009 &     -1.4661 &       0.0040 \\
  $a_{21}$      &     0.0000 &     -0.0011 &     -4.2691 &       0.0016 \\
  $a_{22}$      &     0.2500 &      0.0022 &      3.6085 &       0.0039 \\
  $a_{31}$      &     0.0000 &     -3.6560 &     -0.1081 &     213.8302 \\
  $a_{32}$      &     0.0000 &    -74.1944 &     -3.1560 &     148.6830 \\
  $a_{33}$      &  1,500.0000 &  -219.2507 &     -5.7537 &     241.0047 \\
  $a_{41}$      &     0.0000 &    -41.2545 &     -1.5464 &     168.7276 \\
  $a_{42}$      &     0.0000 &   -174.8974 &     -3.4213 &     323.3108 \\
  $a_{43}$      &     0.0000 &    -32.2724 &     -0.5806 &     351.5762 \\
  $a_{44}$      &  1,500.0000 &  -170.8750 &     -3.5702 &     302.7048 \\
  \midrule
  \mc{1}{l}{Steps}          & 324   & & Evaluations & 1429 \\
  \mc{1}{l}{RMSE}           & 0.0665  & & & \\
  \bottomrule
  \end{tabular}\scriptsize
  \begin{tablenotes}\item \textbf{Notes:} Std. Deviation = Standard Deviation. I calculate the number of steps, number of evaluations, and the RMSE as the average across all forty Monte Carlo iterations.
  \end{tablenotes}

\end{threeparttable}
\end{center}
\end{table}

\begin{table}\onehalfspacing
\begin{center}
\begin{threeparttable}
  \caption{Monte Carlo Exercise, Dataset Two}
  \label{Monte Carlo: Two}
  \begin{tabular}{crrrr}\toprule

  Parameter & True Value & Bias & $t$ - statistic & Std. Deviation \\
  \midrule
  $\alpha_{10}$ &      \phantom{20000}9.2100 &  -0.0013 & -2.3522 &     0.0036 \\
  $\alpha_{11}$ &      0.0400 &      0.0001 &  1.5572 &    0.0003 \\
  $\alpha_{12}$ &      0.0330 &     -0.0001 & -1.8808 &    0.0003 \\
  $\alpha_{13}$ &      0.0005 &     -0.0000 & -1.2201 &    0.0000 \\
  $\alpha_{14}$ &      0.0000 &     -0.0000 & -0.8527 &    0.0003 \\
  $\alpha_{15}$ &      0.0000 &      0.0000 &  1.2649 &    0.0000 \\
  $\alpha_{20}$ &      8.2000 &     -0.0078 & -5.0148 &    0.0098 \\
  $\alpha_{21}$ &      0.0800 &     -0.0002 & -2.2834 &    0.0005 \\
  $\alpha_{22}$ &      0.0670 &      0.0002 &  1.8267 &    0.0006 \\
  $\alpha_{23}$ &      0.0010 &      0.0000 &  4.5655 &    0.0001 \\
  $\alpha_{24}$ &      0.0220 &     -0.0001 & -1.9589 &    0.0005 \\
  $\alpha_{25}$ &      0.0005 &     -0.0000 & -1.4327 &    0.0000 \\
  $\beta_{0}$   &   5,000.0000 &   -21.6427 & -0.5906 &  231.7552 \\
  $\beta_{1}$   &   5,000.0000 &  -170.0311 & -1.7716 &  607.0113 \\
  $\beta_{2}$   &  15,000.0000 &   218.8722 &  3.3175 &  417.2622 \\
  $\gamma_{0}$  &  14,500.0000 &   -41.7072 & -1.2614 &  209.1211 \\
  $a_{11}$      &      0.4000 &     -0.0401 & -1.4301 &    0.1772 \\
  $a_{21}$      &      0.0000 &      0.0051 &  2.8442 &    0.0114 \\
  $a_{22}$      &      0.5000 &     -0.0028 & -1.9463 &    0.0089 \\
  $a_{31}$      &      0.0000 &    -95.9679 & -2.2467 &  270.1549 \\
  $a_{32}$      &      0.0000 &    106.9453 &  1.5877 &  426.0040 \\
  $a_{33}$      &   6,000.0000 &   132.6752 &  3.3306 &  251.9426 \\
  $a_{41}$      &      0.0000 &     14.5530 &  0.4718 &  195.1054 \\
  $a_{42}$      &      0.0000 &    -82.8807 & -1.7011 &  308.1402 \\
  $a_{43}$      &      0.0000 &   -111.7468 & -2.6635 &  265.3480 \\
  $a_{44}$      &   6,000.0000 &   -38.1595 & -1.4635 &  164.9108 \\
  \midrule
  \mc{1}{l}{Steps}          & 21   & & Evaluations &  460\\
  \mc{1}{l}{RMSE}           & 0.0346  & & & \\
  \bottomrule
  \end{tabular}\scriptsize
  \begin{tablenotes}\item \textbf{Notes:} Std. Deviation = Standard Deviation. I calculate the number of steps, number of evaluations, and the RMSE as the average across all forty Monte Carlo iterations.
  \end{tablenotes}
\end{threeparttable}
\end{center}
\end{table}

\begin{table}\onehalfspacing
\begin{center}
\begin{threeparttable}
  \caption{Monte Carlo Exercise, Dataset Three}
  \label{Monte Carlo: Three}
  \begin{tabular}{crrrr}\toprule

  Parameter & True Value & Bias & $t$ - statistic & Std. Deviation \\
  \midrule
  $\alpha_{10}$ &    \phantom{20000}8.0000 &   -0.0094 &  -4.5655 &    0.0130 \\
  $\alpha_{11}$ &      0.0700 &    -0.0000 &  -0.1778 &     0.0016\\
  $\alpha_{12}$ &      0.0550 &     0.0002 &   0.7982 &     0.0017\\
  $\alpha_{13}$ &      0.0000 &    -0.0000 &  -1.4478 &     0.0000\\
  $\alpha_{14}$ &      0.0000 &    -0.0031 &  -4.8617 &     0.0041\\
  $\alpha_{15}$ &      0.0000 &     0.0006 &   4.3643 &     0.0008\\
  $\alpha_{20}$ &      7.9000 &    -0.0022 &  -2.5053 &     0.0056\\
  $\alpha_{21}$ &      0.0700 &    -0.0009 &  -2.8882 &     0.0020\\
  $\alpha_{22}$ &      0.0600 &     0.0010 &   3.4609 &     0.0019\\
  $\alpha_{23}$ &      0.0000 &     0.0000 &   3.0568 &     0.0000\\
  $\alpha_{24}$ &      0.0550 &    -0.0015 &  -2.8318 &     0.0034\\
  $\alpha_{25}$ &      0.0000 &    -0.0001 &  -2.2934 &     0.0002\\
  $\beta_{0}$   &   5,000.0000 &  103.2942 &   0.7202 &   907.1527\\
  $\beta_{1}$   &   5,000.0000 &   14.8701 &   0.1120 &   839.9011\\
  $\beta_{2}$   &  20,000.0000 &  276.1337 &   1.0797 &  1,617.5471 \\
  $\gamma_{0}$  &  21,500.0000 &   19.5391 &   4.3537 &    28.3842\\
  $a_{11}$      &      1.0000 &    -0.0336 &  -4.9505 &     0.0429\\
  $a_{21}$      &      0.5000 &    -0.0112 &  -2.1246 &     0.0335\\
  $a_{22}$      &      0.8660 &    -0.0053 &  -2.2026 &     0.0152\\
  $a_{31}$      &      0.0000 &  -191.8160 &  -1.5377 &   788.9319\\
  $a_{32}$      &      0.0000 &   503.8073 &   1.7409 &  1,830.2752 \\
  $a_{33}$      &   7,000.0000 &  626.6112 &   3.6343 &  1,090.4486\\
  $a_{41}$      &      0.0000 &   254.0053 &   2.0260 &   792.9084 \\
  $a_{42}$      &      0.0000 &  -671.4902 &  -2.7732 &  1,531.3821 \\
  $a_{43}$      &  -4,250.0000 & -440.7818 &  -1.2436 &  2,241.6838 \\
  $a_{44}$      &   7,361.2159 &  418.5595 &   1.5500 &  1,707.8543 \\
  \midrule
  \mc{1}{l}{Steps}          &  17  & & Evaluations & 415 \\
  \mc{1}{l}{RMSE}           & 0.024  & & & \\
  \bottomrule
  \end{tabular}\scriptsize
  \begin{tablenotes}\item \textbf{Notes:} Std. Deviation = Standard Deviation. I calculate the number of steps, number of evaluations, and the RMSE as the average across all 40 Monte Carlo iterations.
\end{tablenotes}
\end{threeparttable}
\end{center}
\end{table}
\clearpage
%-------------------------------------------------------------------------------
\subsection{Interpolation Schemes}
%-------------------------------------------------------------------------------
Table \ref{Interpolation Schemes} shows the estimation results based on alternative interpolation schemes. For the static estimation, I solve the complete DP problem, use 500 random draws for the evaluation of $E\max$, and allow for a maximum of 1,000 evaluations of the criterion function by the optimizer. I use a single processor for all estimations.\begin{table}\onehalfspacing
\begin{center}
\begin{threeparttable}
  \captionsetup{width=30cm}
  \caption{Interpolation Schemes}
  \label{Interpolation Schemes}
  \begin{tabular}{lrrrr}\toprule
  Points      & 200 & 500 & 1500  & All \\
  $E\max$ Draws & 500 &  500 &   500 & 500 \\
  \midrule
  RMSE        & 0.10 &   0.06 &    0.05 &  0.03  \\
  Minutes     &  13 &      70 &    112 &   1995 \\
  Steps       &  573 &   2848 &    2307 &  21415 \\
  Evaluations & 1838 &   6745 &    6327 &  41285 \\
  \bottomrule
  \end{tabular}
  \end{threeparttable}
  \end{center}
\end{table}

%-------------------------------------------------------------------------------
\subsection{Smoothing Parameters}
%-------------------------------------------------------------------------------
Table \ref{Smoothing Schemes} shows the estimation results based on alternative smoothing parameters. For the static estimation, I solve the complete DP problem, use 500 random draws for the evaluation of $E\max$, and allow for a maximum of 1,000 evaluations of the criterion function by the optimizer. I use a single processor for all estimations.\begin{table}\onehalfspacing
\begin{center}
\begin{threeparttable}
  \captionsetup{width=30cm}
  \caption{Smoothing Schemes}
  \label{Smoothing Schemes}
  \begin{tabular}{lrrrr}\toprule
  $\tau$ & 500 &  400 &   300 & 200 \\
  Points      & 200 & 200 & 200  & 200 \\
  \midrule
  RMSE        & 0.10 &   0.06 &    0.05 &  0.03  \\
  Minutes     &  13 &      70 &    112 &   1995 \\
  Steps       &  573 &   2848 &    2307 &  21415 \\
  Evaluations & 1838 &   6745 &    6327 &  41285 \\
  \bottomrule
  \end{tabular}
  \end{threeparttable}
  \end{center}
\end{table}

